\documentclass[12pt,english]{article}
\usepackage{amsmath}
\usepackage{graphicx}
\graphicspath{{./}}

\title{Dishonest or Defensive Discourse Patterns}
\author{Levi Finkelstein}
\date{\today}

\begin{document}
\maketitle
\\\\
\it{This is a list in progress of discourse related behaviors that frequently irk me when I'm talking to people. This can be anything from clear dishonesty, to more unconscious dishonesty resulting from cognitive dissonance, to small  psychological defensive ticks, to vague uninformative unclear statements, to anything else that annoys me. Some of these I'll label with terms I read elsewhere, if I don't know a fitting term I'll just make one up. I will update with examples as I think of them.}
\par
\textbf{Question Dodging} 
\begin{quote}
\textbf{Q:} Do you believe X?\\ \textbf{A:} Well, I don't like X, and X is something I think a lot about. \\ \textbf{Q:} If you learned X would it change your mind on Y? \\ \textbf{A:} My mind changes all the time, and I have my reasons for believing Y, like for example...\\ \textbf{Q:} What would you do in this hypothetical scenario?\\ \textbf{A:} This scenario would never happen.\\ 
\end{quote}
\par
\textbf{Unrealistic Hypothetical}\\
You pose someone a hypothetical question that test the consistency of some principle they claim to hold and they make up excuses to not answer it like it being "unrealistic" or something like that.

\par
\textbf{Ironic Agreement/Answer}\\
Sometimes people will ironically say they "lost the debate" or they'll dodge a question with some obviously ironic voice giving some ridiculous answer.\\
\begin{quote}

Yees! You were SOO right, I'm SOO wrong about this!\\ Of course I believe <thing I obviously don't believe but now I don't have to commit seriously to anything so the discussion can't proceed and I have plausible deniability that I was just trolling all along, and btw did I tell you I don't take this conversation seriously at all>!\\
\end{quote}
I think a lot of people just instinctively resort to this type of irony when they're panicking.
\par
\textbf{Universally Applicable Analogy Rejection}\\
Just saying "They're not the same thing..." in response to an analogy without an actual explanation of how the analogy fails. Of course they're not literally the same thing, that's the whole point of an analogy. The analogy intuition pumps a conclusion you're uncomfortable with so you flinch and say they're not the same thing even though they actually are the same thing. If you get asked how they're not the same thing you just keep repeating "they're not the same thing" over and over. Maybe you even say "How can you compare X and Y? They're completely different things!!"
\par
\textbf{Crux Allergy}\\
A crux is something specific your belief hinges on in such a way that it makes a non-trivial difference to what you believe. Since people can be quite averse to the possibility of having their beliefs falsified they'll often repeatedly weaken their claim every time some concrete crux is identified: (\href{https://www.lesswrong.com/posts/NKaPFf98Y5otMbsPk/bayesian-judo}{source})
\begin{quote}
    \textbf{Tom:} I don't believe Artificial Intelligence is possible because only God can make a soul.\\
    \textbf{Sarah:} You mean if I can make an Artificial Intelligence, it proves your religion is false?\\
    \textbf{Tom:} Well, I didn't mean that you couldn't make an intelligence, just that it couldn't be emotional in the same way we are.\\
    \textbf{Sarah:} So if I make an Artificial Intelligence that, without being deliberately preprogrammed with any sort of script, starts talking about an emotional life that sounds like ours, that means your religion is wrong.\\
    \textbf{Tom:} Well, um, I guess we may have to agree to disagree on this.\\
\end{quote}
\par
\textbf{False Crux}\\
Even though it's difficult to establish cruxes it's preferable to the alternative:
\begin{quote}
    \textbf{John:} I'm really interested in changing your mind on X, why do you believe X?\\
    \textbf{Carl:} I believe X because of Y!\\
    \textbf{John:} \textit{Spends a few hours researching.}\\
    \textbf{John:} Here's irrefutable evidence that Y is false.\\
    \textbf{Carl:} Hmm, well, I still believe X because of Z.\\
    \textbf{John:} ...\\
\end{quote}
Often people don't know why they believe things so if you ask them they'll just make something up on the spot since they're not comfortable acknowledging that they believe something without being able to articulate reasons.
When you falsify the reasons they gave they're emotionally not persuaded since those weren't the real reasons, so they'll just make up some other reasons. This is \textit{partly} why crux allergy is so pervasive: once you get them to imagine a world in where their given reasons are falsified they can more emotionally connect with the fact that in that world their mind wouldn't be changed.
\par
\textbf{}\\

\end{document}
