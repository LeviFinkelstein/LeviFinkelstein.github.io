\usepackage{pgfplots}
\title{Mind-Tracing}
\author{Levi Finkelstein}
\date{27 08 2021}
\begin{document}
\maketitle

Visualize yourself standing in a space familiar enough to you that you could lucidly walk through it with little conscious effort. Choose a starting position in the space that feels natural, go there, then don't move! Now you're ready to begin.
\\\\
\begin{enumerate}
    \item Just focus on the image of yourself standing at your chosen location. Nothing complicated.
    \item  Eventually, as is inevitable, you'll lose focus. Your mind will silently drift toward some other thought. Soon you'll notice this and realize how your inner surroundings have changed. Now comes the trick.
    \item  Before returning to your original location take a snapshot of the train of thought you slipped into. This is just some mental image that should encapsulate the thoughts that distracted you. The image doesn't have to be clear or vivid, it only has to be some token that clearly represents what you were just thinking about.
    \item  Take this image back with you to the starting position of the space you chose in the beginning. Now, place the image beneath your feet, then walk to another location in the space, keeping the image's position and meaning clear in your mind. Now you should be standing at a new location.
    \item Repeat steps 1-4 until you've created a path in the space such that each point in the path contains the mental image that hijacked your thoughts and caused you to move. Make sure that you're always able to re-trace the entire path. Do this for as long as you can, it will become progressively more difficult as the path grows in size.
 
\end{enumerate}
 I call this exercise "mind-tracing". How this compares to meditation I'm not sure, but I find it plausible that it's way better given that the core tenet of meditation (focusing, and re-focusing when thoughts drift) is incorporated in addition to heavy use of visualization and working-memory and giving you a clear overview of the evolution of your thoughts.

\end{document}
