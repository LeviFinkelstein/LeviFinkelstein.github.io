\title{Math Intuition I}
\author{Levi Finkelstein}
\date{qwer}

\begin{document}
\maketitle
Here are some ways to interpret equations.
\par
This is the stupdiest way.
In high school you learn about equations like this: $10 = 4 + 3x^2$.
You might learn to solve it (and all that means to you is that the $x$ will be completely alone on one side of the equal sign) by applying different rules to it, for example you are allowed to move a term to the other side of the equal sign if you change the sign: $10 - 4 = 3x^2$
and also you're allowed to divide by a number if you do it on both sides: $\frac{6}{3}=\frac{3x^2}{3}\iff 2=x^2$ and take the square root on both sides, but if you do it to a variable raised to an even power then you have to put a $\pm$ in front of it: $x = \pm\sqrt{2}$.

\par

This understanding is completely syntactical and involves literally no understanding of why the *rules* follow from the meaning of the symbols.

\par

\end{document}
