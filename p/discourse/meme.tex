\documentclass[12pt,english]{article}
\usepackage{amsmath}
\usepackage{graphicx}
\graphicspath{{./}}

\title{Dishonest or Defensive Discourse Patterns}
\author{Levi Finkelstein}
\date{\today}

\begin{document}
\maketitle
\\\\
\it{This is a list in progress of discourse related behaviors that frequently irk me when I'm talking to people. This can be anything from clear dishonesty, to more unconscious dishonesty resulting from cognitive dissonance, to small  psychological defensive ticks, to vague uninformative unclear statements, to anything else that annoys me. Some of these I'll label with terms I read elsewhere, if I don't know a fitting term I'll just make one up. I will update with examples as I think of them.}
\par
\textbf{Not answering questions.} 
\begin{quote}
\textbf{Q:} Do you believe X?\\ \textbf{A:} Well, I don't like X, and X is something I think a lot about. \\ \textbf{Q:} If you learned X would it change your mind on Y? \\ \textbf{A:} My mind changes all the time, and I have my reasons for believing Y, like for example...\\ \textbf{Q:} What would you do in this hypothetical scenario?\\ \textbf{A:} This scenario would never happen.\\ 
\end{quote}
Gives me a feeling of intellectual cowardice, or non-appreciation for making specific points in response to specific ideas.

\par
\textbf{Not entertaining hypotheticals.}\\
You pose someone a hypothetical that test the consistency of some principle they claim to hold and they make up excuses to not answer it like it being "unrealistic" or something like that.

\par
\textbf{Ironic Agreement/Answer} Sometimes people will ironically say they "lost the debate" or they'll dodge a question with some obviously ironic voice giving some ridiculous answer.\\
\begin{quote}

Yees! You were SOO right, I'm SOO wrong about this!\\ Of course I believe <thing I obviously don't believe but now I don't have to commit seriously to anything so the discussion can't proceed and I have plausible deniability that I was just trolling all along, and btw did I tell you I don't take this conversation seriously at all>!\\
\end{quote}
I think a lot of people just instinctively resort to this type of irony when they're panicking.
\par
\textbf{Universally Applicable Analogy Rejection}\\
Just saying "They're not the same thing..." in response to an analogy without an actual explanation of how the analogy fails. Of course they're not literally the same thing, that's the whole point of an analogy. The analogy intuition pumps a conclusion you're uncomfortable with so you flinch and say they're not the same thing even though they actually are the same thing. If you get asked how they're not the same thing you just keep repeating "they're not the same thing" over and over. Maybe you even say "How can you compare X and Y? They're completely different things!!"
\par
\textbf{In Progress...}

\end{document}
