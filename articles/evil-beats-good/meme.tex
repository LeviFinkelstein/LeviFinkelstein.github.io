\usepackage{pgfplots}
\title{Evil always wins over Good (all else equal)}
\author{Levi Finkelstein}
\date{27 08 2021}
\begin{document}
\maketitle

I was reading about this trick used for proving that a greedy algorithm is optimal called "greedy exchange". It works by showing that any solution produced by an optimal algorithm can be morphed into the solution produced by your greedy algorithm without destroying its optimality.
\\\\
Reading about this trick reminded my of the reason for why Evil will always win over Good: \textbf{anything a Good agent does an Evil agent can do, but not vice versa}.
\\\\
If $U$ is the set of all actions and $E\in U$ is the set of Evil actions, then the Evil agent has the advantage of being able to search through all of $U$ for actions, while the Good agent is constrained to the smaller subset $U/E\in U$. 
\\\\
Mr. Nice cannot act optimally relative to Mr. Mean since where Mr. Nice wouldn't torture someone for his goal, Mr. Mean would. Mr. Mean is solving a relaxed version Mr. Nice's problem, and Mr. Nice is solving a constrained version of Mr. Mean's problem.
\\\\
In some sense this doesn't matter at all. If Mr. Nice and Mr. Mean are just two agents with each their own objective function, what does it mean to say that the one is more constrained than the other, aren't they both just optimizing action for their objective functions? I think yes. The sense in which this principle applies is if you think of Evil in a deontological sense where you imagine superimposing deontological rules like "don't lie" onto the space through which you're searching for actions. In other words deontology works against you maximizing your objective function. Which is in my eyes the reason why deontology doesn't make any sense. If I have a deontological rule that says I can't do X, then I can just always be faced with the scenario of "all sentient life being tortured for all eternity unless I do X" and I would be fucked.
\\\\
The principle applies however in the sense of: the social ladder climber that's willing to lie, manipulate, murder, etc. will outclimb the social ladder climber who's not willing to do those things. They have the same goal, but one search space is more relaxed than the other. However, in practice it might not be the case that the Evil climber always outclimbs on average, for example because with a limited mind it might be very difficult to figure out exactly when and how to act in an Evil way where the benefits would outweigh the grave consequences you'll face if you commit a mistake.
\\\\
So who cares if Evil wins if "Evil" just means not alligning with some stupid deontology nonsense? In this sense of Evil I don't think we should care, but there is another sense in which it matters. The sense in which optimizing for Good might not optimize you for optimizing.

%TODO continue about how that deontology nonsese makes it so peole who maximize power will outcompete people who maximize Good which is only partly overlapping with power. Objective function of power. If you only search for power you can spend time feeling good. Myopic dystopia. that moloch thing

\end{document}
